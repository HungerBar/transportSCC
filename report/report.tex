\documentclass[final]{beamer}

% ------------------------------
% Poster size: A1 landscape
% ------------------------------
\usepackage[size=a1,orientation=landscape,scale=1.25]{beamerposter}

% ------------------------------
% Packages (Chinese packages removed)
% ------------------------------
\usepackage{graphicx}
\usepackage{multicol}
\usepackage{amsmath}
\usepackage{enumitem}
% Removed: \usepackage{xeCJK}
% Removed: \setCJKmainfont{SimHei}

\begin{document}

\setmainfont{Arial} % Keep Arial for English text

% Increase text sizes for poster readability
\setlength{\parindent}{0pt}
\setlength{\parskip}{0.6em}
\newcommand{\PosterSection}{\vspace{0.8em}\LARGE\bfseries}

% ------------------------------
% Modern Dark Blue Theme
% ------------------------------
\definecolor{PosterBlue}{HTML}{083D77}
\definecolor{PosterLight}{HTML}{F4F7FA}

\setbeamercolor{background canvas}{bg=PosterLight}
\setbeamercolor{block title}{fg=white,bg=PosterBlue}
\setbeamercolor{normal text}{fg=black}


% -----------------------------------------------------------
% Title Banner (Dark Blue)
% -----------------------------------------------------------
\begin{frame}[t]
\begin{beamercolorbox}[wd=\paperwidth,ht=6cm,center,dp=2cm]{title}
    {\veryHuge\bfseries
    Butterfly Model of the WWW and Its Application to China's Transportation Network}
\end{beamercolorbox}

\vspace{1.5cm}

% -----------------------------------------------------------
% 3-column layout
% -----------------------------------------------------------
\begin{multicols}{3}

% ===========================================================
% Column 1
% ===========================================================

\PosterSection Broder et al. (2000) "Bow-Tie" Model

{\Large
\begin{itemize}[leftmargin=1.2em]
    \item WWW can be decomposed into SCC, IN, OUT, Tendrils, Tubes, Islands.
    \item SCC extremely dense: any two pages reach each other in < 20 clicks.
    \item IN, OUT, and SCC each occupy \textbf{24\%–28\%} of all nodes.
    \item Peripheral structures have sparse or directional connectivity.
\end{itemize}
}

\PosterSection Applications

{\Large
\begin{itemize}[leftmargin=1.2em]
    \item Network robustness analysis
    \item Information diffusion modeling
    \item Optimization \& vulnerability detection
    \item Influential in social networks, citation graphs, finance
\end{itemize}
}

\PosterSection Data \& Redefinition

{\Large
\begin{itemize}[leftmargin=1.2em]
    \item Data: China domestic flights + railway (OpenStreetMap)
    \item Constructed directed transport network
    \item Classical SCC too large due to dense infrastructure
    \item New definitions:
    \begin{itemize}
        \item SCC: in/out degree $> k$
        \item Island: degree $< m$
    \end{itemize}
    \item Here: $k=42,\ m=6$
\end{itemize}
}

\begin{figure}[h]
\centering
\includegraphics[width=0.90\linewidth]{map1.png}
\caption*{\Large Transportation Network (placeholder)}
\end{figure}


% ===========================================================
% Column 2
% ===========================================================

\PosterSection SCC / Island Classification

\begin{figure}[h]
\centering
\includegraphics[width=0.95\linewidth]{map2.png}
\caption*{\Large SCC vs Island under $(k=42,m=6)$}
\end{figure}

\PosterSection Subnetwork Visualization

\begin{figure}[h]
\centering
\includegraphics[width=0.95\linewidth]{map3.png}
\caption*{\Large Subcomponents (IN/OUT/others)}
\end{figure}

\PosterSection Observations

{\Large
\begin{itemize}[leftmargin=1.2em]
    \item Macau classified as Island → data reliability to investigate
    \item SCC cities show high structural centrality
    \item Removing any SCC city reduces global connectivity by \textbf{>30\%}
\end{itemize}
}


% ===========================================================
% Column 3
% ===========================================================

\PosterSection Implications

{\Large
\begin{itemize}[leftmargin=1.2em]
    \item Bow-tie framework reveals macro topology
    \item Useful for national-scale robustness evaluation
    \item Helps design optimal resilience strategies
\end{itemize}
}

\PosterSection Connectivity Loss

\begin{figure}[h]
\centering
\includegraphics[width=0.95\linewidth]{damage_ratio.png}
\caption*{\Large SCC Reduction vs Damage Ratio}
\end{figure}

\PosterSection Damage Impact

\begin{figure}[h]
\centering
\includegraphics[width=0.95\linewidth]{bar_damage.png}
\caption*{\Large Impact of Individual SCC Node Removal}
\end{figure}

\PosterSection TODO / Future Work

{\Large
\begin{itemize}[leftmargin=1.2em]
    \item Identify ideal $k$ threshold
    \item Validate outlier cities under alternative datasets
    \item Incorporate multi-modal \& temporal networks
\end{itemize}
}

\end{multicols}
\end{frame}

\end{document}
